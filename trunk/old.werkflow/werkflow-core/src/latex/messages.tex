\chapter{Messages}

\section{Overview}

\emph{Messages} represent the primary way that a
WFMS gains knowledge about the environment so that it
may react appropriately.  Some messages may cause new
procces cases to be initiated, while others may apply
to a particular existing case.  

{\werkflow} maintains a deliberately abstract concept
of messages internally.  Through the extension components
of \emph{message selectors}, \emph{message correlators},
and \emph{messaging managers}, the WFMS can remain
oblivious to the actual form of messages used by the application.

\section{Message Types}

A \emph{message type} is a uniquely differentiable type
of message expected by a process.  Differentiation is accomplished
through collaboration between a \emph{message selector} and
an compatible \emph{messaging manager}.  

\section{Message Selectors}

A \emph{message selector}, held by a \emph{message type}, 
collaborates with a \emph{messaging manager} to differentiate
messages and route them to the appropriate point in the process.

A selector is intimately tied to the messaging manager in that it
is purely opaque to the WFMS and only intepreted by the messaging
manager.  For example, a JMS-centric message selector could be
used by a JMS-centric messaging manager to perform topic registration.
The message-type would then identify messages from a particular
subscription.

An example using the \verb|werkflow:basic| syntax:

\begin{codelisting}
<message-type id="po">
    <jms:selector topic="topic://purchase-orders"/>
</message-type>
\end{codelisting}


\section{Messaging Manager}

\section{Message Correlators}

