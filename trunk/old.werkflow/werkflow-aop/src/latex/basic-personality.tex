% ----------------------------------------------------------------------
%     The Basic Personality
% ----------------------------------------------------------------------

\chapter{The Basic Personality}

% ----------------------------------------------------------------------
%     Introduction
% ----------------------------------------------------------------------

\section{Introduction}

The \emph{basic personality} mimics many of the structures
of other \emph{block-based} syntaxes, such as BPML and BPEL4WS
but does not enforce the same semantic constraints.  The syntax of
the basic personality is fairly procedural and contains many common
flow-control structures.

The basic personality's syntax is accessed through XML tags from
the \verb|werkflow:basic| namespace.

\begin{codelisting}
<process \hired{xmlns="werkflow:basic"}
         package="my.package"
         id="my.process">
</process>
\end{codelisting}

% ----------------------------------------------------------------------
%     Prolog
% ----------------------------------------------------------------------

\section{Prolog}

Each process definition begins with a common prolog that includes
the \tag{process} tag and attribute definitions for the process.

\subsection{\tag{process}}

The \tag{process} tag begins a process definition. It requires an
\verb|id| attribute to uniquely identify the process.  It may also
contain a \verb|package| attribute for defining the process within
the scope of a larger package.  The \verb|initiation| attribute is
required in order to specify how new cases are initiated.

\begin{tabular}{l|c|c|l}
\textbf{attribute}   &	\textbf{type}		&	\textbf{default}	& \textbf{description} \\
\hline
package		&	optional	&			& Package scope identifier. \\
id			&	required	&			& Unique identifier for the process. \\
initiation	&	required	&			& Initiation type. \verb|call| or \verb|message|.
\end{tabular}

\begin{codelisting}
\hired{<process package="my.package"
         id="my.process"
         initiation="message">}
    \dots
\hired{</process>}
\end{codelisting}

\subsection{\tag{attributes}}

Each process contains an enumeration of \emph{case attributes} which
describe the data carried and used by each case.  The \tag{attributes}
tag is simple a container tag for \tag{attributes} tags.

\begin{codelisting}
<process package="my.package"
         id="my.process"
         initiation="message">

    \hired{<attributes>}
         \dots
    \hired{</attributes>}
</process>
\end{codelisting}

\subsection{\tag{attribute}}

% ----------------------------------------------------------------------
%     Flow-Control Structures
% ----------------------------------------------------------------------

\section{Flow-Control Structures}

\subsection{\tag{sequence}}

\subsection{\tag{parallel}}

\subsection{\tag{if}}

\subsection{\tag{while}}

\subsection{\tag{switch}}

\subsection{\tag{pick}}


The \tag{if} tag

% ----------------------------------------------------------------------
%     Actions
% ----------------------------------------------------------------------

\section{Actions}




