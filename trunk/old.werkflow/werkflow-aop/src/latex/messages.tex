\chapter{Messages}

\section{Overview}

\emph{Messages} represent the primary way that a
WFMS gains knowledge about the environment so that it
may react appropriately.  Some messages may cause new
procces cases to be initiated, while others may apply
to a particular existing case.  

{\werkflow} maintains a deliberately abstract concept
of messages internally.  Through the extension components
of \emph{message selectors}, \emph{message correlators},
and \emph{messaging managers}, the WFMS can remain
oblivious to the actual form of messages used by the application.

\section{Message Types}

A \emph{message type} is a uniquely differentiable type
of message expected by a process.  Differentiation is accomplished
through collaboration between a \emph{message selector} and
an compatible \emph{messaging manager}.   By using 
message selectors, the semantic type of the message may be
completely unrelated to the underlying Java type of the
message objects.

\section{Message Selectors}

A \emph{message selector}, held by a \emph{message type}, 
collaborates with a \emph{messaging manager} to differentiate
messages and route them to the appropriate point in the process.

A selector is intimately tied to the messaging manager in that it
is purely opaque to the WFMS and only intepreted by the messaging
manager.  For example, a JMS-centric message selector could be
used by a JMS-centric messaging manager to perform topic registration.
The message-type would then identify messages from a particular
subscription. 

Here is an example using the \verb|werkflow:basic| syntax.  This
scenarios assumes that the topic from which messages arrive is
sufficient to differentiate them.

\begin{codelisting}
<message-type id="po">
    <jms:selector topic="topic://purchase-orders"/>
</message-type>
\end{codelisting}

A process using the \tag{jms:selector} may only be deployed
against a messaging manager capable of correctly interpreting
the \verb|jms:selector| 

An XML-centric message selector may inspect the the root
element's name and namespace of each document to determine
if it is selected.  Each message may be of the same underlying
Java class, such as \verb|org.dom4j.Document|, but the
semantic type is still differentiable.

\begin{codelisting}
<message-type id="po">
    <xml:selector name="purchase-order" uri="http://myco.com/po.xsd"/>
</message-type>
\end{codelisting}

The \tag{java:selector} allows for differentiating messages by
their Java class and a filtering expression.  

\begin{codelisting}
package com.myco.messages;

public class PoMessage
\{
    private String type;
    private String poId;

    public PoMessage(String type,
                     String poId)
    \{
        this.type = type;
        this.poId = poId;
    \}

    public String getType()
    \{
        return this.type;
    \}

    public String getPoId()
    \{
        return this.poId;
    \}
\}
\end{codelisting}

\begin{codelisting}
<message-type id="accept.po">
    <java:selector type="com.myco.messages.PoMessage"
                   filter="message.getType().equals( &quot;accept&quot; )"/>
</message-type>
<message-type id="accept.po">
    <java:selector type="com.myco.messages.PoMessage"
                   filter="message.getType().equals( &quot;reject&quot; )"/>
</message-type>
\end{codelisting}

\section{Messaging Manager}

A \emph{messaging manager} is the conduit for injecting
messages into the WFMS.  Each message type used by a deployed
process is registered against the messaging manager.  The
messaging manager uses the associated message selector to
arrange for matching messages to the deployed process's
\emph{message sink}.

The message selector \emph{must} be of a type expected
by the messaging manager, otherwise an exception noting the
incompatibility may be thrown, preventing the deployment
of the process.  {\werkflow} does not define how messages
are generated or received external to it.  It only registers
its interest with the messaging manager through the message
type and message sink objects.  It is then the resposibility
of the messaging manager to route messages appropriately.

\section{Message Correlators}

With the exception of \emph{initiation messages}, 
any message sent to the WFMS by a messaging manager must
be correlated with an existing in-progress case.  For example,
a \emph{Results of Credit Check} message received
by a process should be associated with the case that initiated
the request.  The act of matching an incoming message with
an existing case is \emph{message correlation} and is
controlled through \emph{message correlators}.

A message correlator is responsible for comparing a new
inbound message with an existing case and determining if
they match.  It is routine to maintain attributes in the
case useful for correlation.  For example, the results of
a credit-check may include the bank's loan application file 
id as a reference.  

\begin{codelisting}
package com.myco.messages;

public interface CreditCheckResults
\{
    String getLoanApplicationFileId();
    int getCreditScore();
\}
\end{codelisting}

\begin{codelisting}
<process id="loan.app">
    <attributes>
        <attribute id="appFileId">
            \dots
        </attribute>
    </attributes>
    <message-types>
        <message-type id="credit.results">
            <java:selector tyep="com.myco.messages.CreditCheckResults"/>
        </message-type>
    </message-types>
    \dots
</process>
\end{codelisting}

Correlation can occur between the \verb|getLoanApplicationFileId()|
results and the \verb|appFileId| attribute.  Correlation may also
involve multiple attributes and complex logic.  Each message
type may correlate differently.

Here is an example, using the \verb|werkflow:basic| personality, of 
receiving a correlated message:

\begin{codelisting}
\dots
<receive id="credit.results">
    <jexl:correlator test="message.loanApplicationFileId == appFileId"/>
    \dots
</receive>
\dots
\end{codelisting}

Using this, each case that is waiting to \tag{receive} a
\emph{credit.results} message will continue to wait
not just for a \emph{credit.results} message, but
one whose result of the \verb|getApplicationFileId()|
method is equal to the case attribute \verb|appFileId|.

