\chapter{Jelly Semantic Module}

\section{Overview}

Jelly is an XML scripting framework originally created by
James Strachan and maintained by The Apache Software Foundation's
Jakarta project.  The Jelly language is infinitely extendable
through custom \emph{tag libraries}.  Many tag libraries
already exist for a variety of activities, such as a database
manipulation, sending email, manipulating JMS topics, invoking
web services, and even executing jakarta-ant tasks.

It is simple to add additional libraries at runtime using
\emph{XML Namespace} declarations.  The Jelly semantic
module itself is imported into process definitions using
the \verb|werkflow:jelly| namespace URI.

Jelly uses \verb|\$\{...\}| notation to signal expressions
to be evaluated, typically using the Jexl expression language.

\begin{codelisting}
<process \hired{xmlns:jelly="werkflow:jelly"} \dots>
    \dots
\end{codelisting}


\section{\tag{jelly:action}}

The \tag{jelly:action} tag defines a block of Jelly script
to be executed \emph{atomically} and thus can be used
as the action for message reception.  It may also be used
in any context that other constructs are allowed.

\begin{codelisting}
<sequence>
    <receive \dots>
        \hired{<jelly:action xmlns:j="jelly:core" 
                      xmlns:http="jelly:http">
            <j:set var="caseAttrOne" value="\$\{message.propOne\}"/>
            <j:set var="caseAttrTwo" value="\$\{message.propTwo\}"/>
            <http:get \dots/>
        </jelly:action>}
    </receive>
</sequence>
\end{codelisting}


