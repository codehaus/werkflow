% ----------------------------------------------------------------------
%     Basic Personality
% ----------------------------------------------------------------------

\chapter{Basic Personality}

% ----------------------------------------------------------------------
%     Introduction
% ----------------------------------------------------------------------

\section{Introduction}

The \emph{basic personality} mimics many of the structures
of other \emph{block-based} syntaxes, such as BPML and BPEL4WS
but does not enforce the same semantic constraints.  The syntax of
the basic personality is fairly procedural and contains many common
flow-control structures.

The basic personality's syntax is accessed through XML tags from
the \verb|werkflow:basic| namespace.

\begin{codelisting}
<process \hired{xmlns="werkflow:basic"}
         package="my.package"
         id="my.process">
</process>
\end{codelisting}

Tags from other namespaces may also be used in order to 
take advantage of the various semantic modules, such
as \verb|werkflow:java|, \verb|werkflow:jelly| and 
\verb|werkflow:python|

% ----------------------------------------------------------------------
%     Prolog
% ----------------------------------------------------------------------

\section{Prolog}

Each process definition begins with a common prolog that includes
the \tag{process} tag and attribute definitions for the process.
Also, any message types that the process expects to receive must
be declared.

\subsection{\tag{process}}

The \tag{process} tag begins a process definition. It requires an
\verb|id| attribute to uniquely identify the process.  It may also
contain a \verb|package| attribute for defining the process within
the scope of a larger package.  The \verb|initiation| attribute is
required in order to specify how new cases are initiated.

\begin{codelisting}
\hired{<process package="my.package"
         id="my.process"
         initiation="message">}
    \dots
\hired{</process>}
\end{codelisting}

\begin{attrDefs}
package		&	optional	&			& Package scope identifier. \\
id			&	required	&			& Unique identifier for the process. \\
initiation	&	required	&			& Initiation type. \verb|call| or \verb|message|.
\end{attrDefs}

\subsection{\tag{message-types} and \tag{message-type}}

Each process contains an enumeration of the \emph{message types}
which it expects to receive.  The \tag{message-types} tag is a simple
container tag for \tag{message-type} tags.

\begin{codelisting}
<process package="my.package"
         id="my.process"
         initiation="message">

    \hired{<message-types>}
         \dots
    \hired{</message-types>}
</process>
\end{codelisting}

Each message-type that is used by the process is defined using
a \tag{message-type} tag.  An \verb|id| attribute is required
to uniquely identify each type.  Each \tag{message-type} 
contains a \emph{message selector} as its content.  The
message selector must be one that is appropriate and compatible
with the \emph{messaging manager} in use, such as
\tag{java:selector}.

\begin{codelisting}
<message-types>
    \hired{<message-type id="publish">}
        \dots
    \hired{</message-type>}
</message-types>
\end{codelisting}

\begin{attrDefs}
id		&	required	&			& Unique identifier for the message-type.
\end{attrDefs}

\subsection{\tag{attributes} and \tag{attribute}}

Each process contains an enumeration of \emph{case attributes} which
describe the data carried and used by each case.  The \tag{attributes}
tag is simple a container tag for \tag{attribute} tags.

\begin{codelisting}
<process package="my.package"
         id="my.process"
         initiation="message">

    \hired{<attributes>}
         \dots
    \hired{</attributes>}
</process>
\end{codelisting}

Each attribute that is maintained for each process case is defined
using an \tag{attribute} tag.  It requires an \verb|id| attribute to
uniquely identify the attribute.  Optional \verb|in| and \verb|out|
boolean attributes determine if the attribute is use for input 
and/or output parameter passing.

\begin{codelisting}
<attributes>
    \hired{<attribute id="attrOne">}
        \dots 
    \hired{</attribute>}
</attributes>
\end{codelisting}

\begin{attrDefs}
id			&	required	&			& Unique identifier for the attributes. \\
in			&	optional	& ``false''	& Input parameter flag. \verb|true| or \verb|false| \\
out			&	optional	& ``false''	& Output parameter flag. \verb|true| or \verb|false| \\
\end{attrDefs}

Each \tag{attribute} tag must contain an \emph{attribute type} 
as the content, such as \tag{java:attr-type}.


% ----------------------------------------------------------------------
%     Flow-Control Structures
% ----------------------------------------------------------------------

\section{Flow-Control Structures}

The flow of the process affects the order in which activities can be
executed.  The two main structures are \tag{sequence} and
\tag{parallel} which denote items that must be executed sequential
and those that may be executed in parallel.  The \tag{while} looping
construct allows for repeating a segment until a condition is met,
while \tag{if}, \tag{swtich} and \tag{pick} allow for condition
execution of segments.

\subsection{\tag{sequence}}

The \tag{sequence} tag has no attributes and denotes that its
content should be executed in a sequential mannager.  Each element
will only execute after the previous element has completed.

\begin{codelisting}
\hired{<sequence>}
    \dots
    \dots
\hired{</sequence>}
\end{codelisting}

\subsection{\tag{parallel}}

The \tag{parallel} tag has no attributes and denotes that
its context \emph{may} be executed in parallel.  The actual
execution order may differ for each case, and the body may
not actually be executed in a fully concurrent manner.

\begin{codelisting}
\hired{<parallel>}
    \dots
    \dots
\hired{</parallel>}
\end{codelisting}


\subsection{\tag{while}}

The \tag{while} tag allows for the execution of its content
repeatedly while a loop condition evaluates to \verb|true|.
The \verb|condition| attribute is a \emph{Python} expression
evaluated within the context of the case's attributes.

\begin{codelisting}
\hired{<while condiiton="status == 'editing'">}
    \dots
    \dots
\hired{</while>}
\end{codelisting}

\begin{attrDefs}
condition	&	required	&			& Python loop expression. \\
\end{attrDefs}

\subsection{\tag{if}}

The \tag{if} tag allows for conditional execution of its
content if a test condition evaluates to \verb|true|.
The \verb|condition| attribute is a \emph{Python} expression
evaluated within the context of the case's attributes.


\begin{codelisting}
\hired{<if condiiton="status == 'rejected'">}
    \dots
    \dots
\hired{</if>}
\end{codelisting}

\begin{attrDefs}
condition	&	required	&			& Python loop expression. \\
\end{attrDefs}

\subsection{\tag{switch}, \tag{case} and \tag{otherwise}}

The \tag{switch} tag allows for the execution of exactly one of
many branches. It has no attributes and directly contains only
\tag{case} tags.  Each \tag{case} tag contains a test condition
to determine if its branch should be executed.  The \tag{case}
tags are evaluated in order and the first to evaluate to
\verb|true| is executed, skipping all others.  The \tag{switch}
may optionally contain, as the last content tag, an \tag{otherwise} tag
that contains no attributes and is executed if no other branches
were selected.

\begin{codelisting}
\hired{<switch>
    <case condition="status == 'editing'">}
        \dots
        \dots
    \hired{</case>
    <case condition="status == 'rejected'">}
        \dots
        \dots
    \hired{</case>
    <otherwise>}
        \dots
        \dots
    \hired{</otherwise>
</switch>}
\end{codelisting}

\subsubsection{\tag{case} attributes}

\begin{attrDefs}
condition	&	required	&			& Python loop expression. \\
\end{attrDefs}

\subsection{\tag{pick}}

The \tag{pick} tag allows for the execution of exactly one
of many branches determined by the type of message received,
or a time-out (note, time-out not implemented yet).  It has
no attributes and directly contains only \tag{choice} tags.
Each \tag{choice} tag has no attributes.  The body of each
\tag{choice} must have, as the first effective action, 
either a \tag{receive} or a \tag{wait}.

\begin{codelisting}
\hired{<pick>
    <choice>}
        <sequence>
            \hired{<receive type="rejection.message"
                \dots
            </receive>}
            \dots
            \dots
        </sequence>
    \hired{</choice>
    <choice>}
        <sequence>
            \hired{<receive type="acceptance.message"
                \dots
            </receive>}
            \dots
            \dots
        </sequence>
    \hired{</choice>
    <choice>}
        <sequence>
            \hired{<wait \dots/>}
            \dots
            \dots
        </sequence>
    \hired{</choice>
</pick>}
\end{codelisting}


% ----------------------------------------------------------------------
%     Actions
% ----------------------------------------------------------------------

\section{Atomic Actions}

\subsection{\tag{receive}}

The \tag{receive} tag specifies that the process should wait
for a correlated message of a particular type.  It has a \verb|type|
attribute that refers to a previously-defined message type.
When the \tag{receive} action is an initiating action, it may
only contain a single atomic action as the body to handle the
reception and integration of the message.   An optional \verb|bind|
attribute may specify the name to which the message is bound for
the context of the action. When used in other
contexts to cause an already existing process to wait for a message,
the body may also contain a \emph{correlator} such as 
\tag{python:correlator}.

\begin{codelisting}
\hired{<receive type="some.message">}
    <jexl:correlator test="message.someProp == someCaseAttr"/>
    <jelly:action>
        <j:set var="caseAttrOne" value="\$\{message.propOne\}"/>
        <j:set var="caseAttrTwo" value="\$\{message.propTwo\}"/>
    </jelly:action>
\hired{</receive>}
\end{codelisting}

\begin{attrDefs}
type	&	required	&				& Message type identifier \\
bind	&	optional	& ``message''	& Message binding variable \\
\end{attrDefs}

\subsection{\tag{terminate}}

The \tag{terminate} tag immediately terminates the process
case.  It has not attributes.

\begin{codelisting}
\hired{<terminate/>}
\end{codelisting}

\subsection{Actions From Other Semantic Modules}

For actions beyond \tag{receive}, other semantic modules such
as \verb|werkflow:jelly| and \verb|werkflow:java| can be used
to perform the actual work.  Please see their documentation 
elsewhere.

