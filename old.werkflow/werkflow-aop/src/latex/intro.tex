
\chapter{Introduction}

\section{``Everything is workflow''}

\emph{Workflow} is a term generally used to describe systems that
assist users in the accomplishment of so process.  A workflow 
system may be something as simple as a printed-and-bound book
of standard operating proceduces, or it may encompass many
different information systems potentially across several
enterprise boundaries.

Workflow has typically been applied to ``human-scale'' procedures
that have tradionally been performed by shuffling paper files 
and forms between people.  Workflow, though, is not inherently
limited to these types of activities.  The theory underlying
workflow is applicable to procedures of many different scales
and domains.  For example, the negotiation of a TCP could be
considered to be workflow.

\section{Processes, Cases, and Attributes}

A \emph{process} is the description of how to acommplish a goal.
Typically a process describes some countable or quanitfiable
activity.  \emph{Loan officer} does not describe a countable
process, but \emph{Handle Loan Application} does.  In a given
day, a bank can count the number of loan applications it
started handled, finished handling, are or still in-flight.
Ideally the number of actual loan officers remains roughly
static over time.  Processes also benefit from a sense
of hierarchy.  The \emph{Build Workflow Engine} process may
include a hierarchy of \emph{Implement Feature}, \emph{Fix Bug},
and \emph{Ship A Release} sub-processes.

A process merely describes how to do something, but each
time the process is started a new \emph{process case} is
created to track the status of that instance of the 
process.  For instance, the worker who accepts a loan
application from a customer may create a new manilla
folder with the application and send it to the next
person who must work with it.  In this case, the folder
acts as the unifying case.  Each case may have information
associated with it that accumlates or changes over time
In a workflow system, these are called \emph{case attributes}.
The actual loan application would be an attribute, as
would be the results of an initial credit-check performed
by someone in the credit department.

\section{Workflow Management Systems}

Just as data management was extracted from applications
and relocated in database management system, workflow
management is starting to be extracted to \emph{workflow
management systems} or \emph{WFMS}.  A WFMS, when
loaded with a set or process definitions, helps determine
when to start a new case, what should be done next, and
when a case is complete.  A WFMS that is used for human-scale
processes typically models a digital \emph{inbox} where
human works can inspect and operate on tasks offered to them
by the system.  A lower-level WFMS targetting \emph{web-service
choreography} may not interact at all with human workers, but
rather performs all steps of the process automatically.
In general, a WFMS helps decide what can or should be
done for each case, and many times the WFMS itself can
arrange for the work to be performed without human
intervention, in a fully automated fashion.

A completel WFMS provides not just the engine that drives
the execution of each case, but also process definition 
authoring tools, runtime administration tools for deploying
processes, runtime metrics gathering and analysis tools and other
peripheral components for functions such as clustering and
integration to external systems.

Specializations, such as BPEL4WS and BPML exist to narrow
both the problem domain and the solution domain and target
a domain such as web-services.  These mostly fall into the
category of \emph{choreography} which may certainly
choreograph the activites of a workflow.  

So, everything is a workflow, but not everything is the
\emph{same type} of workflow.

\section{{\werkflow} is a WFMS}

The {\werkflow} project aims to be a complete WFMS with
several key components released as open-source, while making
others available on a commercially licensed basis.  The core
rutime engine is available under a license similar to the
Apache Software License which allows inclusion in commercial
products.

{\werkflow} has been designed as an abstract process execution
engine upon which higher semantic layers such as \emph{BPEL4WS}
may be applied.  The {\werkflow} team noted that the biggest
difference between the various types of werkflow was syntactic.
Underneath, they all express similar structures and models.

