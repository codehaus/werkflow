\chapter{Processes}

\section{Instructions}

Foremost, a process describes \emph{what to do}.  The
instructions of a process maybe a simple linear list, such as
a recipe.  It could be a non-sequential set of activities to
do, in no particular order.  Most times, though, the instructions
of a process contain both a mixture of sequential and non-sequential
activities.  They also contain conditional logic that may
only be required in certain circumstances.

\section{Attributes}

Each process works with a particular set of data.  The data
may be used while performing activities or maybe be used to 
determine if a conditional set of activities should be
performed.  The attributes typically include primary keys
use to locate business objects in external systems.

\section{Initiation}

To create a new case for a process, the process must be
\emph{initiated}.  Initiation can be a manual activity
or may occur in response to some initiating message.

*** need to expand all of this ***


