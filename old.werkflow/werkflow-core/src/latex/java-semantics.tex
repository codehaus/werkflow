\chapter{Java Semantic Module}

\section{Overview}

The Java semantic module uses the BeanShell library to allow
Java expressions and scripts to be used in process definitions.
The Java tag library is imported using an XML namespace
declaration with the URI of \verb|werkflow:java|.

\begin{codelisting}
<process \hired{xmlns:java="werkflow:java"} \dots>
    \dots
\end{codelisting}

\section{\tag{java:selector}}

The \tag{java:selector} tag allows selection of messages based
upon their true Java class and an optional filtering expression.
This selector only works when the WFMS is configured to use a 
\verb|SimpleMessagingManager|.  

Each message object received by the messaging manager will subject 
to an \verb|instanceof| check against the class specified by 
the \verb|type| attribute.  If an optional filtering expression
is provided by the \verb|filter| attribute, it is evaluated
with the message bound to the \verb|message| variable.

\begin{codelisting}
<message-type \dots>
    \hired{<java:selector type="com.myco.messages.PoMessage"
                   filter="message.getType().equals( &quot;reject&quot; )"/>}
</message-type>
\end{codelisting}

\begin{attrDefs}
type		&	required	&			& Java message class name. \\
filter		&	optional	&			& Filtering Java expression. \\
\end{attrDefs}

\section{\tag{java:correlator}}

The \tag{java:correlator} tag takes a single \verb|test| attribute
that should contain a boolean Java expression determining if
the message matches the process case.  The containing message-reception
action may specify that the message is bound to a specific variable
name.  If not otherwise specified, the message object will be
bound to the \verb|message| variable. 

\begin{codelisting}
<receive \dots>
    \hired{<java:correlator test="message.getPoId().equals( poId )"/>}
</receive>
<receive \hired{bind="command"} \dots>
    \hired{<java:correlator test="command.getAction().equals( &quot;accept&quot; )"/>}
</receive>
\end{codelisting}

\begin{attrDefs}
test		&	required	&			& Java boolean correlation expression. \\
\end{attrDefs}

\section{\tag{java:action}}

The \tag{java:action} tag allows for Java logic to be used
as an action in several ways.

\subsection{Inline Java}

The \tag{java:action} tag may contain a block of BeanShell Java
to be interpreted as an atomic action.  The process case attributes
and other action attributes are bound to similarly named variables
within the block.

The language dialect used in that which is supported by BeanShell
and thus is not pure Java.  BeanShell provides various syntactic
additions to make scripting easier.  \emph{http://beanshell.org/}
provides a complete reference.

\begin{codelisting}
\hired{<java:action>}
    import java.util.Date;

    System.err.println( new Date() + " :: accepted po:" + poId );
\hired{</java:action>}
\end{codelisting}

\subsection{JavaBeans}

The \tag{java:action} tag may also be used to invoke a method
on typical Java class. It may invoke any method with two
\verb|Map| parameters on any accessible Java class.  When
used this way, the tag requires a \verb|type| attribute
that names the class to be used as the action.  The class
must contain a public default constructor.  An optional
\verb|method| attribute may specify the name of the method 
to invoke.  If not otherwise specified, the default method 
name is \verb|execute|.

\begin{codelisting}
package com.myco.actions;

import java.util.Map;

public class MyRandomBean
\{
    public void execute(Map caseAttrs,
                        Map otherAttrs)
        throws Exception
    \{
        \dots
    \}

    public void doIt(Map caseAttrs,
                     Map otherAttrs)
        throws MySpecialException
    \{
        \dots
    \}
\}
\end{codelisting}

This \tag{java:action} invoked the 
\verb|execute(...)| method on the \verb|MyRandomBean|:

\begin{codelisting}
\hired{<java:action type="com.myco.actions.MyRandomBean"/>}
\end{codelisting}

This \tag{java:action} invoked the 
\verb|doIt(...)| method on the \verb|MyRandomBean|:

\begin{codelisting}
\hired{<java:action type="com.myco.actions.MyRandomBean"
             method="doIt"/>}
\end{codelisting}

\begin{attrDefs}
type		&	required	&			& Java class name. \\
method		&	optional	& ``execute'' & Method name to invoke. \\
\end{attrDefs}
